\chapter{Background}
\label{ch:background}

Formal methods are a specific type of mathematical notation which is based on the techniques of the specification, verification and development of software and hardware systems \cite{whatareformalmethods}. Since our thesis presents \gls{zmath} we go right to the beginning of the framework, to describe how mathematical notation came about. Then we describe the original MathLang framework (the framework which \gls{zmath} is an adaptation of) and then give the reader an idea of other formal methods and languages. In the next section we wish to describe what is the language of Z and give more details of it's syntax and semantics. We then highlight other proving techniques which have been done for maths, formal methods and Z.

\section{Mathematical Notations}

Computer science (and thus computer systems) have evolved from basic mathematics. We can say that formal specification writers are practicing mathematicians as they write system specifications in a formal manner. Therefore we must start right at the beginning at the foundation of mathematical notation.

\subsection{Right from the begininning}

The relationship between mathematical reasoning and practicing mathematicians started out early on during the ancient Greeks where logic was already being studied. Reasoning in logic was used for just about anything not just mathematics such as law, medicine and farming. This very early form of mathematics made very famous discoveries such as Aristotles logic \cite{aristotle}, Euclid's geometry \cite{euclid} and Leibniz Calculus \cite{leibniz}.

Further on in the 1800's, Frege wrote \emph{Die Grundlagen der Arithmetik} \cite{frege} and other works where he noted that mathematics is a branch of logic. In this works, he began building a solid foundation for mathematics. This early foundation along with Cantors set theory \cite{cantor} was argued to be incosistant and thus Russel found a pardox in this work.

In the late 19th century and beginning of 20th century, Russell \& Whitehead \cite{whitehead1912principia} started to form a basis for mathematical notation. Their three volume work describes a set of rules from which all mathematical truths could be proven. In these early stages the authors try to dervive all maths from logic. This ambitious project was the first stepping stone in collaborating all mathematics under one notation.

Further to Russell \& Whitehead's work, Bourbanki\footnote{A name given to a collective of mathematicians} wrote a series of books beginning n the 1935's with the aim of grounding mathematics. Their main works is included in the Elements of Mathematics series \cite{opac-b1128208} which does not need any special of knowledge of mathematics. It describes mathematics from the very beginning and goes through core mathematical concepts such as set theory, algebra, function etc and gives complete proofs for these concepts.

Adding to Russell's work, Zermelo introduced an axiomatisation of set teory which was later extended by Frankel and Skolem to form ZF set theory \cite{zfc}. This new theory is what we will later see the Z notation is based on and the notation this thesis checks the correctness of.

\subsection{Computerisation of Maths and Proof Systems}

In the 21st century, a great area of research is how use, store and support this mathematical knowledge. Since automation has become more and more used, mathematicians have looked into ways in which they can use computers to reason about and provide services to mathematics. This would include all areas of mathematics, such as logic, mechanics and software specifications. Mathematical knowedge can be represent in lots of different ways including the following:

\begin{itemize}
\item Typesetting systems like \LaTeX{}

\item proof assistants and automated theorem provers

\item Semantical oriented document representations like OpenMath and OMDoc
\end{itemize}


\subsection{Conclusion}

In summary....

\section{MathLang for mathematics}
\label{sec:mathlang}

Intro....

\subsection{Overview and Goals}

\subsection{Detailed information on CGa}

\begin{itemize}
\item Reference Zenglars quote

\item Weak type theory into CGa

\end{itemize}

\subsection{Detailed information on DRa}

\begin{itemize}
\item relations

\item instances

\item Dependency and goto graph
\end{itemize}

\subsection{Detailed information on skeletons}

\begin{itemize}
\item General Proof Skeleton

\item Half baked proof

\item Filled in skeleton
\end{itemize}

\subsection{information on TSa}

TSa is used to formalise mathematical texts, mathematicians write differently eg `a=b=c` is the same as `a=b and b=c`. We do not need tsa as formal spec are already written formally.

\subsection{A full worked examples in mathlang}

show step by step translation of mathematical text into isabelle from laamars phd thesis.

\subsection{Conclusion}

\section{Formal Methods and Languages}
\label{sec:formalmethodsandformallanguages}

\begin{itemize}
\item definitions of `formal language', `formal method' and `formal specification'

\item the first formal language is thought to be used by Frege in his Begriffsschift (1879), Begriffschift meaning `concept of writing' described as `formal language of pure thoguht'

\item broad histroy of formal methods 

\begin{itemize}
\item 1940's, Alan Turing annotated the properties of program states to simplify the logical analysus of sequential programs

\item 1960's Floyd, Hoare and Naur recommended using axiomatic techniques to prove programs meet their specification.

\item 1970's Dijkstra used formal caluculus to aid development of non-deterinist programs
\end{itemize}

\item Formal methods today

\item why use formal methods in industry (design errors like Therac-25 1985, NASA’s Checkout Launch and Control System (CLCS) cancelled 9/2002, , added level of rigor)

\item types of formal methods (Z, B method, ABS)

\item Success of formal methods (B27 Traffic Control System, SHOLIS project, Data Acquisition, Monitoring and Commanding of Space Equipment)

\item Weakness of formal methods (Low-level ontologies, Limited Scope,Cost,Poor tool feedback)

\item What needs to be done to make “formal methods” industrial strength? 
\begin{itemize}
\item Bridge gap between real world and mathematics
\item Mapping from formal specifications to code (preferably automated)
\item Patterns identified
\item Level of abstraction should be supported
\item Tools needed to hide complexity of formalism
\item Provide visualization of specifications 
\item Certain activities not yet ‘formulizable’ methods
\item No one model has been identified which should be used for software)
\end{itemize}
\end{itemize}

\subsection{Conclusion}

\section{Z Syntax and semantics}
\label{sec:theznotation}

\subsection{Why Z would work with MathLang}

Bridges formal method and discrete mathematical notation.

\subsection{Introduction to Z}

Z is based on predicate Calculus, Zermelo-Frankel set theory...

Ivented by j-R Abriel, ISO standard.+ Spivey standard

\subsection{Propositional and predicate logic}

\subsection{Sets and Types}

\subsection{Definitions}

\begin{itemize}
\item AxDef
\item Freetypes
\item Schema (declarations and expressions)
\end{itemize}

\subsection{A full example in Z}


\subsection{Conclusion}

\section{Proving systems for Z}

Intro....

\subsection{Levels of Rigor}

\begin{itemize}
\item Level 1 represents the use of mathematical logic to specify the system.
\item Level 2 uses pencil-and-paper proofs.
\item Level 3 is the most rigorous application of formal methods.

\end{itemize}

\subsection{Proving systems for maths}

e.g. Mizar, Isabelle, Coq

\subsection{Proving systems for formal method}

e.g. Dafny, ALC2, PVS

\subsection{Proving Systems specific for Z}

e.g. Fuzz, Hol-z ProofPower-z

\subsection{Other proeprties to prove}
\label{subsec:propertiestoprove}

\subsection{Conclusion}

\section{Background Conclusion}

\subsection{MathLang for Z}

\begin{itemize}
\item \cite{fmpresetation} states what ro do to make formal methods industrial strength

\item \cite{lamarphd} stating in future work mathlang should be developed to cope with more mathematics (formal spec is a type of mathematics)

\item diagram of math text to theorem prover using mathlang + diagram of specification to theorem prover using mathlang
\end{itemize}

ZMathLang covers items 1, 3, 5, 6, 7 from section \ref{sec:formalmethodsandformallanguages}.