
\chapter{Introduction}
\label{ch:introduction}

Industries developing high integrity software are always looking for ways to make their software safer. Safety Integrity Levels (SILs) are used to define a level of risk-reduction provided by a safety-function. The highest SIL which could be given to hardware or safety integrity, as given by the International Electro technical Commission's (IEC) standard, is a SIL4. SIL4 has a probability of failure of 0.0001-0.00001\%, although this is very low since it is still not at 0\% which means there is always room for improvement. Software testing usually takes place when the program or a prototype has been implemented. However by the time the product is fully implemented and errors are caught it is expensive to go back to the planning stage to find solutions to those bugs. Catching errors at an earlier stage of the project life cycle would be more time and cost effective for the whole project team.

One idea to detect errors at an early stage of the project life cycle is by applying the use of \gls{fm} at the design/specification stage of the project life cycle. The benefit of using \gls{fm} is that they provide a means to symbolically examine the entire state space of a design and establish a correctness that is true for all possible inputs \cite{wifrm}. However due to the enormous complexity of real time systems they are rarely used. \Gls{fm} come in different shapes and sizes; the \gls{asm} theory is a state machine which operates on states or arbitrary data structures. The B-method \cite{bmeth} is a formal method for the development of program code from a specification in the \gls{asm} notation. Z \cite{spiveyreferencemanual} is a specification languages used for describing computer-based systems. \Gls{uml} provides system architects with a consistent language for specifying, construing and documenting different components of systems. These are just a selection of various \gls{fm} methods however there are a great deal more which are still applied to systems today to add a degree of safety to certain high integrity products.

Specifications models and verification may be done using different levels of rigour. Level 1 represents the use of mathematical logic to specify a system, level 2 uses a handwritten approach to proofs and level 3 is the most rigorous application of formal methods which uses theorem provers to undertake fully formal machine-checked proofs. Level 3 is the most expensive out of the three levels and only practically worthwhile when the cost of making mistakes is extremely high.

Since the jump from Level 1 to a Level 3 level of rigour is very difficult, this thesis introduces an approach where the large jump is broken up in to multiple smaller jumps.

\section{Motivations}

In order to facilitate the \gls{comp} process, this thesis proposes smaller \gls{comp} steps which allow the translation (and hence the correctness checking) of specifications written in Z into a theorem prover such as ProofPower-Z \cite{pp} or Isabelle \cite{isabelle}.
The reason to break the translation path into simple easier steps is because the original path is difficult and requires serious expertise in theorem proving and the translation from Z to a theorem prover. My approach of mini-\gls{comp} steps allows different experts to collaborate on the various steps to build the final proof.

The list below provides the main motivations to the research of this thesis:

\section{Contributions}

A summary of contributions is given in the following points:

\begin{itemize}
\item ZMathLang's \gls{zcga} has been created and implemented.

\begin{itemize}

\item Weak types and weak typing rules have been thoroughly implemented. 

\item A style file has been created to label a specification with \gls{zcga} annotations. This style file also outputs coloured boxes around weak types in the specification so that the user can see the weak types in a clear manner.

\item A weak type checker, which reads the \gls{zcga} annotations and checks them has been implemented.

\item Examples given for various specifications \footnote{The examples for various specifications can be found on http://www.macs.hw.ac.uk/~lb89/zmathlang/examples}.
\end{itemize}

\item ZMathLang's \gls{zdra} has been created and implemented. Dependency Graphs and Goto Graphs have been implemented to be automatically generated from the \gls{zdra} annotated document.

\begin{itemize}
\item Instance names and relations have been carefully realized and added to the \LaTeX{} style file.

\item Relation rules have been outlined.

\item The \gls{zdra} has been implemented to check for the document rhetorical correctness which outputs various warning and error messages.

\item Using directed graphs, dependency and GoTo graphs can be automatically generated from the implementations. Formal aspects of these graphs have also been highlighted.

\item Examples given for various specifications.
\end{itemize}

\item \gls{gpsa} and Isabelle skeletons have been implemented so they are automatically generated from the \gls{zdra} annotated document.

\begin{itemize}

\item Using the Goto graph a general proof skeleton can be automatically created from the implementation.

\item Using the general proof skeleton an Isabelle skeleton and a filled in skeleton of the original specification can be automatically generated using the implementation.

\item A formal definition of the \gls{zdra}, Dependency graph and GoTo graph has been given.
\end{itemize}

\item A gradual computerisation path from Z specifications (BirthdayBook, Vending Machine and Ed Curries book) have been documented. These are the first translations from raw Z specifications to complete proofs done using the \gls{zmath} framework. Out of these translations we get new dependency graphs and proof skeletons for the individual specifications.

\begin{itemize}
\item Clear and concise translation paths from various Z specifications into \gls{zcga} annotated and checked documents, \gls{zdra} annotated and check documents.

\item Dependency graphs, GoTo Graphs and general proof skeletons are generated for all example specifications.

\item Isabelle skeletons automatically generated and filled in for all example specifications.

\item Safety property's and lemma's added to example specifications and proved.
\end{itemize}

\end{itemize}

\section{Outline}

In chapter \ref{ch:background} we begin describing the origins of \gls{math}, it's success and where it has been used so far. We then go on to describe Z specifications and the tools available for it so far.

In chapter \ref{ch:design}, we outline the basic idea of \gls{zmath}, how the original \gls{math} method has been adapted to perform with Z specifications and how this method is different to others previously described.

Chapter \ref{chap:zcga} gives great details the first contribution of this thesis. The weak types which have been created are presented as well as how they work together with weak typing rules. The categories which have been extracted from the weak types are presented and examples are given on how these categories correspond to Z specifications. Examples are given for all the weak types, and categories for Z specification. The weak type checker which is implemented in python is thoroughly described and details of how the tool can be used is given.

Chapter \ref{ch:zdra} highlights another aspect which is a contribution of this thesis. An explanation of what rhetorical correctness for a specification is given. Instances and how they relate to each other are described as well as how a user can annotated these facts into a Z specification. Examples are given for all relations and instances and rules of what relations are allowed. An outline of the \gls{zdra} checker is given, along with explanations of various error and warning messages. A general explanation of the dependency and GoTo graphs are also given.

In chapter \ref{ch:skeletons} describes the different skeletons which can be automatically generated if the specification is \gls{zdra} correct. A detail explanation of how a general proof skeleton can be created from the GoTo graph is given along with the algorithm which creates it. A summary is given of how the general proof sketch can be used to generate an Isabelle skeleton of the specification. Details of how the Isabelle skeleton can be filled in using the \gls{zcga} annotated specification is also shown in this chapter. A demonstration of how we can use this filled in Isabelle skeleton to get a full proof is also described.

Chapter \ref{ch:formal} gives formal definitions of the \gls{zdra}, dependency and GoTo skeletons. We prove various properties about the \gls{zdra} using rules and definitions which have been given. We give examples of how each of these aspects can be represented in a formal manner. The algorithm which creates the dependency and GoTo is given and explained.

In chapter \ref{ch:interface}, we give an overview on the Interface design. We explain how one can run and use \gls{zmath} on Z specifications. Explanations of how to use each aspect is given via examples and screen shots. The tables of output messages which a user can receive are highlighted and explained.

Chapter \ref{ch:fullexample} goes through one entire specification (vending machine) along the \gls{zmath} route. Each aspect is clearly highlighted and explained to the reader giving hints and tips along the way. Other examples are found in the appendix however they are only taken along the \gls{zmath} route without any commentary.

In chapter \ref{ch:analysis} we look at two specification examples which have been proved in a theorem prover using one step and compare it with the space specification examples which have been prove in multiple steps using \gls{zmath}. We give a table of comparison explaining the amount of expertise required, type of input and lines of proofs and lemmas. We explain and compare the type of expertise required for each of the specification examples and how they compare against doing the proof in one step or multiple steps.

Finishing the thesis, chapter \ref{ch:conclusion} the summarises the work and contributions of the research. We give a final discussion and reflect back on the developments. We reflect upon the benefit and limitations of the research and conclude with explanations on current and future work.
