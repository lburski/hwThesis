\chapter{Conclusion and Future Work}
\label{ch:conclusion}

In this chapter we discuss the current development of \gls{zmath} and it's future works. We also conclude a comparisson between \gls{zmath} framework to other ststems. Finally in section we give add concluding thoughts to this thesis.

\section{ZMathLang Current and Future Developments}
\label{sec:zmathcurandfut}

\subsection{Current Developmets}
\label{subsec:currendevelopments}

The research on ZMathLang was started in 2013 and provides a novice approach to translating Formal specification to theorem provers. With this approach the gradual translation of the formal specification document is made via "aspects". Each aspect checks for a different type of correctness of the formal specification and output different products in order to analyse the system. Moreover, the annotation of the formal specification document should not require any expertise skills in the language of the targatted theorem prover. The only expertise needed for the annotations include the expertise of the formal specification document.

The ground basis of the \gls{math} framework were studied by Maarek, Retel, Laamar and various other master and undergraduate students under the supervision of F.Kamareddine and J.B. Wells. This thesis presents the ground basis of the \gls{zmath} framework which uses the methodology of the \gls{math} framework. The \gls{zmath} framework has taken the idea of breaking up the translation path from a document into a theorem prover and taking it through a grammar correctness checker, a rhetorical correctness checker, a skeleton into a proof. All the theory and implementation of the \gls{zmath} aspects have been developed and described in this thesis.

\subsubsection{Other Developments}

An extension to \gls{zmath} has started being developed by Fellar \cite{zmathmaster}, \cite{ozmathconference} which takes the concept of \gls{zmath} and adds object orientatedness to it. With this, \gls{zmath} has the potential to translate not only Z specifications but object-Z specifications as well. 

This thesis presents a very basic user interface to use with \gls{zmath}. Further developments on the user interface has been expanded during an internship by Mihaylova \cite{zmathuser}, \cite{zmathinternship}. The expansion on the user interface allows users to load and write their specifications. As well as going through each of the correctness checks, viewing the various graphs and skeletons all in one screen.

\subsection{Future Developments}
\label{subsec:futuredevelopments}

The future developments of \gls{zmath} 

\subsubsection{Automisation of the annotation}

\subsubsection{Extension to more complex proof obligations}

\subsubsection{Any formal specification to any theorem prover}

\subsubsection{Informal specifications}

\section{Related Works}
\label{sec:relatedworks}

\section{Conclusion}
\label{sec:conclusion}