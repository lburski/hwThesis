\chapter{Background}
\label{ch:background}

Introduction stating formal methods are a type of mathematics. Explanation of where formal languages came from in mathematics etc

\section{Mathematical Notations}

Intro....

\subsection{Right from the begininning}

\begin{itemize}
\item logic and aristotle

\item Frege Grundgesetze + Cantor + Russel discovered paradoxes

\item Russel inventing type theory

\item Zermelo added axiomisation

\item Fraenkel + skolem extended to ZF set theory (which Z is based on)

\end{itemize}

\subsection{Computerisation of Maths and Proof Systems}

\begin{itemize}
\item typesetting systems like \LaTeX{}

\item proof assistants and automated theorem provers

\item Semantical oriented document representations like OpenMath and OMDoc
\end{itemize}


\subsection{Conclusion}

In summary....

\section{MathLang for mathematics}

Intro....

\subsection{Overview and Goals}

\subsection{Detailed information on CGa}

\begin{itemize}
\item Reference Zenglars quote

\item Weak type theory into CGa

\end{itemize}

\subsection{Detailed information on DRa}

\begin{itemize}
\item relations

\item instances

\item Dependency and goto graph
\end{itemize}

\subsection{Detailed information on skeletons}

\begin{itemize}
\item General Proof Skeleton

\item Half baked proof

\item Filled in skeleton
\end{itemize}

\subsection{information on TSa}

TSa is used to formalise mathematical texts, mathematicians write differently eg `a=b=c` is the same as `a=b and b=c`. We do not need tsa as formal spec are already written formally.

\subsection{A full worked examples in mathlang}

show step by step translation of mathematical text into isabelle from laamars phd thesis.

\subsection{Conclusion}

\section{Formal Methods and Languages}
\label{sec:formalmethodsandformallanguages}

\begin{itemize}
\item definitions of `formal language', `formal method' and `formal specification'

\item the first formal language is thought to be used by Frege in his Begriffsschift (1879), Begriffschift meaning `concept of writing' described as `formal language of pure thoguht'

\item broad histroy of formal methods 

\begin{itemize}
\item 1940's, Alan Turing annotated the properties of program states to simplify the logical analysus of sequential programs

\item 1960's Floyd, Hoare and Naur recommended using axiomatic techniques to prove programs meet their specification.

\item 1970's Dijkstra used formal caluculus to aid development of non-deterinist programs
\end{itemize}

\item Formal methods today

\item why use formal methods in industry (design errors like Therac-25 1985, NASA’s Checkout Launch and Control System (CLCS) cancelled 9/2002, , added level of rigor)

\item types of formal methods (Z, B method, ABS)

\item Success of formal methods (B27 Traffic Control System, SHOLIS project, Data Acquisition, Monitoring and Commanding of Space Equipment)

\item Weakness of formal methods (Low-level ontologies, Limited Scope,Cost,Poor tool feedback)

\item What needs to be done to make “formal methods” industrial strength? 
\begin{itemize}
\item Bridge gap between real world and mathematics
\item Mapping from formal specifications to code (preferably automated)
\item Patterns identified
\item Level of abstraction should be supported
\item Tools needed to hide complexity of formalism
\item Provide visualization of specifications 
\item Certain activities not yet ‘formulizable’ methods
\item No one model has been identified which should be used for software)
\end{itemize}
\end{itemize}

\subsection{Conclusion}

\section{Z Syntax and semantics}

\subsection{Why Z would work with MathLang}

Bridges formal method and discrete mathematical notation.

\subsection{Introduction to Z}

Z is based on predicate Calculus, Zermelo-Frankel set theory...

Ivented by j-R Abriel, ISO standard.+ Spivey standard

\subsection{Propositional and predicate logic}

\subsection{Sets and Types}

\subsection{Definitions}

\begin{itemize}
\item AxDef
\item Freetypes
\item Schema (declarations and expressions)
\end{itemize}

\subsection{A full example in Z}


\subsection{Conclusion}

\section{Proving systems for Z}

Intro....

\subsection{Levels of Rigor}

\begin{itemize}
\item Level 1 represents the use of mathematical logic to specify the system.
\item Level 2 uses pencil-and-paper proofs.
\item Level 3 is the most rigorous application of formal methods.

\end{itemize}

\subsection{Proving systems for maths}

e.g. Mizar, Isabelle, Coq

\subsection{Proving systems for formal method}

e.g. Dafny, ALC2, PVS

\subsection{Proving Systems specific for Z}

e.g. Fuzz, Hol-z ProofPower-z

\subsection{Conclusion}

\section{Background Conclusion}

\subsection{MathLang for Z}

\begin{itemize}
\item \cite{fmpresetation} states what ro do to make formal methods industrial strength

\item \cite{lamarphd} stating in future work mathlang should be developed to cope with more mathematics (formal spec is a type of mathematics)

\item diagram of math text to theorem prover using mathlang + diagram of specification to theorem prover using mathlang
\end{itemize}

ZMathLang covers items 1, 3, 5, 6, 7 from section \ref{sec:formalmethodsandformallanguages}.