\chapter{Conclusion and Future Work}
\label{ch:conclusion}

In this chapter we discuss the current development of \gls{zmath} and it's
future works. We also conclude a comparison between \gls{zmath} framework to
other system. Finally in section \ref{sec:conclusion} we give add concluding thoughts
to this thesis.


\section{Achievements of this thesis}

At the beginning of this thesis we described the motivations and aims of this
thesis these are summarized by the following points:

\begin{enumerate}

\item Staged an approach to translating semi-formal and formal specifications 
into Isabelle with automatic assistance.

\item Built a collection of tools to enable a step by step approach for Z
specifications.

\item Formalised and proved properties on a number of examples of Z
specifications.

\item Demonstrated and evaluated the performance of the tool set on a convincing
 set of examples.

\end{enumerate}

\subsection{Staged an approach to translating semi-formal and formal
specifications into Isabelle with automatic assistance.}

The first accomplishment of this thesis is also the general aim of this thesis.
The step by step method is outlined in chapter \ref{ch:design}, which outlines
how a user can get from a Z specification to a full proof in Isabelle. An
example of this on a single specification is given in chapter
\ref{ch:fullexample}. Each of these steps are described individually throughout
this thesis. There are 6 steps to achieve a full proof for the specification in
question. The first 2 steps require user input and automation, the last step
requires user input and 3 steps in between are fully automated. By following
this method it is easier to translate a specification into a theorem prover with
no theorem prover knowledge up to step 5 (as described in chapter
\ref{ch:analysis}). A \LaTeX{} ZMathLang package has also been created in order
to display ZMathLang annotations when a \LaTeX{} specification is compiled. This
\LaTeX package is also what is used by the ZCGa and ZDRa python program in order
to parse through the specification. The ZMathLang \LaTeX package is shown in
appendix \ref{app:zmathlatex}.

However the limitation of this is that step 5 to step 6 requires user input and
this stage requires some theorem prover knowledge. Proving lemma's in a theorem
prover is not easy and requires expertise in the chosen theorem prover. Apart
from the theorem provers own help tools (such as sledgehammer in Isabelle),
future work may include investigating how to help users with this final stage.
For example automating a way to show users which tactics they may find useful in
proving a certain lemma. Another limitation of this outcome is that even though
the user doesn't need Isabelle expertise to translate their specification into
Isabelle they still need to learn the \gls{zmath} framework. This limitation can
be aided with a user friendly interface and well documented guides such as
\cite{zmathuser}.

\subsection{Built a collection of tools to enable a step by step approach for Z
specifications.}

The second achievement is outlined in chapter \ref{ch:design} and described in detail
in chapter \ref{ch:zcga}. The weak type checker can check for grammatical
correctness of formal and semi formal specification. A \LaTeX{} package named
\texttt{zmathlang.sty} has been implemented which allows the user to annotated
their specification in weak typing categories. When the document is compiled the
annotations then output coloured boxes around each of the categories in their
colours which can be visually analysed by the user. An automatic weak type
checker has been implemented to parse through the specification with it's
annotation to check if the specification is correct or not. The automatic weak
type checker follows a set of rules described in chapter \ref{ch:zcga}.

One limitation of the \gls{zcga} checker is that the user needs to annotate
their specifications by hand using the \LaTeX{} package. This may sometimes be a
repetitive and boring task and improvement to this limitation is described in
section \ref{subsubsec:automisation}. Another restriction to this point is that
although the \gls{zcga} can weakly type semi formal specifications it can only
check the parts which are written in a formal syntax. For example a
\textit{declaration} must be written in the form `\texttt{variable:type}' for
the weak type checker to parse it. A more beneficial weak type checker would
possible be able to parse over \textbf{informal} specifications. More on this
idea is described in section \ref{subsubsec:informalspecs}.

Another point to the second achievement of this thesis was to create a document rhetorical
checker which is described in detail in chapter \ref{ch:zdra}. The document
rhetorical checker can check for any loops in the reasoning in the dependency
and goto graph of the specification. The annotations for the \gls{zdra} are
implemented in \texttt{zmathlang.sty} which can be used on the specification to
annotate chunks of the specification. When using this package to compile the
document, boxes around each of the instances of the specification are shown and
be analysed by the user. An automatic \gls{zdra} program then parses through the
annotations and checks the specification if it is \gls{zdra} correct. Similarly
to the \gls{zcga} checker, the \gls{zdra} checker is implemented in
\texttt{Python}.

Like the \gls{zcga}, the \gls{zdra} annotations for the specification has to be
done by the user. An more user friendly way to do this task would be a
drag-and-drop idea where the user can highlight a piece of specification and
click a button to add what instance this is. The relationships of the \gls{zdra}
could be done in a similar way. This could added to the current user interface
described in chapter \ref{ch:interface}. A second limitation of the current
\gls{zdra} is that users can chunk any part of specification (formal or
informal) the translation from the \gls{zdra} annotated text can only be done
from Z into Isabelle. It may be useful to translate from any formal
specification into Isabelle (or any other theorem prover). More information on
this extension is described in section \ref{subsubsec:anyformanythe}.

The third creation of this research is automatically produce documents which
will be used to aid system engineers and software developers in analysing their
system specifications.There is in total 5 items automatically produce in
\gls{zmath}.

\begin{itemize}
\item dependency graph

\item goto graph

\item \gls{gps}

\item isabelle skeleton

\item \gls{half}
\end{itemize}

The first 4 are automatically produced and stem from a \gls{zdra} correct
specification. The \gls{half} can be automatically produced from a specification
which is both \gls{zcga} and \gls{zdra} correct.

The dependency graph and goto graph (chapter \ref{ch:zdra}) show how the
instances are related to each other, the \gls{gps} (chapter \ref{ch:skeletons})
show in which logical order the instances should be in order to be translated
into a theorem prover with added instance to act as proof obligations. The
Isabelle skeleton (chapter \ref{chap:gpsa2isa}) uses the \gls{zdra} instance
names and creates a skeleton in Isabelle syntax. The \gls{half} (chapter
\ref{chap:gpsa2isa}) is produced by using the Isabelle skeleton and the
\gls{zcga} annotated document. The filled in Isabelle skeleton is therefore the
original specification translated in Isabelle syntax along with added proof
obligations.

One limitation of the \gls{half} is that not all mathematical Z syntax is
translated into Isabelle using \gls{zmath}. The syntax which is translated is
shown in table \ref{tab:latextoisabelle} in chapter \ref{chap:gpsa2isa}. The
current syntax covers all the examples which are in the appendix and in
\cite{mathlangexamples}. However the syntax for all of mathematics is large and
more work can be done on translating more complex mathematical syntax into
Isabelle in \gls{zmath}. These can include schema hiding, piping, conditional
expressions, Mu-expressions \cite{zrefcard} etc.

The proof obligations created in the \gls{gps} are properties which check the
consistence of the specification. These proof obligations are examples of
properties which the user may wish to prove about the specification. Other
complex proof obligations could also be added to \gls{zmath}, more details on
this topic are described in section \ref{subsubsec:extenproofobli}.

\subsection{Formalised and proved properties on a number of examples of Z
specifications.}

The third contribution to this thesis is shown in chapter \ref{ch:fullexample},
chapter \ref{ch:evaluation}, appendix \ref{app:vm6}, \ref{app:bb6},
\ref{app:moduleregfullproof} and in \cite{mathlangexamples} (as all the
examples are quite long and including them will exceed the thesis limit). The ModuleReg
example has been discussed from a raw specification to a fully proven
specification with it's automatically generated lemmas. The lemma's have then
been proven in the theorem prover Isabelle using various tactics. All our
specification examples have been fully proven for consistency checks in Isabelle
except for Vending Machine and Autopilot as they didn't have any State
Invariants and therefore no lemma's where automatically generated for them.

One limitation of this contribution is that the lemma's we have proven
only check the sanity and consistency of the specifications (see figure
\ref{fig:ptp} for definitions) and there are many more checks one can do on a
specification such as properties across specifications and emergent properties
in one specification. One popular way to prove a specification is to create proofs
between a specification and a refinement specification. However a refinement
specification would be difficult to automate on it's own. One way of extended
the ZMathLang toolkit would be to create lemma's between a specification and
it's refinement.

\subsection{Demonstrated and evaluated the performance of the tool set on a convincing
 set of examples.}

 The final contribution to this thesis is we have analysed and evaluated the
 performance of the ZMathLang toolset. This contribution is discussed in
 chapters \ref{ch:analysis} and \ref{ch:evaluation}. We have also discussed
 three case studies in chapter \ref{ch:evaluation} of how the toolkit works. The
 analysis discusses the difference between fully proven specifications using
 other methods Vs specifications proven using ZMathLang. The evaluation discusses
 the complexity of the specification and the advantages of using the ZMathLang
 toolkit as well as some of it's limitations. The complexity of the
 specifications as well as the complexity of the proofs are evaluated when using
the ZMathLang toolkit.

A limitation of this contribution is that the evaluation only looked at the
specifications and their proofs. Other ways to evaluate the specification could
include the time it took to create the proofs and compare them to the time it
took to complete the proof using another method. However this would be quite
difficult as we would need to experiment using the same specification to do a
fair comparison. In some examples we have used, they have not been proved using
another method. The examples which have been proven using a different method
(such as BirthdayBook in ProofPower-Z) are difficult to get figures for (such as
how long the proof would take) as the ProofPower-Z community is no longer
supported.
Further work using the ZMathLang toolkit could include conducting experiments
with theorem prover experts and participants not proficient with theorem proving
and get their feedback on which method they prefer to prove specifications.

\section{Hypothesis and research questions}

At the start of this thesis we asked
\textbf{`to what extent is breaking up the translation path from a
formal/semi-formal specification into a theorem prover, with various
safety checks along the way useful?'}
To help us answer this, we asked the following research questions:

\begin{enumerate}
    \item What and how many steps do we need to break up the translation path?
    \item Which steps can be automated to make it less labour-intensive for the user?
    \item What are the safety checks we can do along the way to a fully proven specification?
    \item Where could a toolkit such as \gls{zmath} be applied?
\end{enumerate}

We can now answer the research questions as follows:

\begin{enumerate}
    \item In chapter \ref{ch:design} we designed the step by step translation
    into 6 steps. We have chosen 6 steps as we tried to follow similar steps as
    the \gls{math} framework for mathematics. This seemed to worked quite well
    as we managed to automate a lot of the translation path using these steps.
    If there were more steps then the jumps from one aspect to another may be
    too large and the user wouldn't be able to digest what was happening. If
    there were more steps then the translation could take too long as there
    would be too much work to do in between. The 6 steps allow for the
    specification to be analysed by various members of the project team. However
    the most \textbf{useful} amount of steps would depend on the project/project
    team and specification in question. Each step has it's own benefits (as
    outlined in each corresponding chapter) therefore none of the steps are expendable.

    \item The overview in chapter \ref{ch:design} and detailed in chapters
    \ref{ch:zcga} to \ref{chap:gpsa2isa} we have managed to automate the
    \gls{zcga} and \gls{zdra} correctness checks (after they have been manually
    annotated). The \gls{gpsa}, Isabelle skeleton, and filled-in Isabelle
    skeleton have also been automated. The last step to prove the specification
    has to be done manually and thus the user would need some theorem prover
    knowledge. It is useful to have the first couple of steps automated as they
    proved to be the most practical according to the comments given by
    industrial experts (written in \ref{ch:evaluation}).

    \item We have written what type of safety checks can be done on a
    specification in chapter \ref{ch:background}. However the safety checks we
    decided to do on our thesis was a weak type check, ZDRa check and proof
    obligations to check the sanity of the specification. These safety checks
    were noted to be quite useful to the industrial experts as written in
    chapter \ref{ch:evaluation}. In particular the Document Rhetorical
    correctness was identified to be helpful in order to identify
    `\textit{knock-on effects}' to solutions in system designs.

    \item The \gls{zmath} toolkit is aimed to be used for `\textbf{high
    integrity}' systems. The Traffic Signal Engineers' review in chapter
    \ref{ch:evaluation} mentions that the \gls{zmath} framework would be useful
    when autonomous vehicles are introduced onto the road. That is, when more
    complex solutions are integrated with current solutions. Since autonomous
    vehicles are still a hot-topic for research there is still time to analyse
    the requirements of such systems thus a toolset such as \gls{zmath} still
    has time to be applied. However, \gls{zmath} would probably not be ideal for
     systems which require a quick turn around.
\end{enumerate}


\section{ZMathLang Current and Future Developments}
\label{sec:zmathcurandfut}

\subsection{Other Current Developments}
\label{subsec:currendevelopments}

The research on ZMathLang was started in 2013 and provides a novice approach to
translating Formal specification to theorem provers. With this approach the
gradual translation of the formal specification document is made via "aspects".
Each aspect checks for a different type of correctness of the formal
specification and output different products in order to analyse the system.
Moreover, the annotation of the formal specification document should not require
any expertise skills in the language of the targeted theorem prover. The only
expertise needed for the annotations include the expertise of the formal
specification document.

The ground basis of the \gls{math} framework were studied by Maarek, Retel,
Laamar and various other master and undergraduate students under the supervision
of F.Kamareddine and J.B. Wells. This thesis presents the ground basis of the
\gls{zmath} framework which uses the methodology of the \gls{math} framework.
The \gls{zmath} framework has taken the idea of breaking up the translation path
from a document into a theorem prover and taking it through a grammar
correctness checker, a rhetorical correctness checker, a skeleton into a proof.
All the theory and implementation of the \gls{zmath} aspects have been developed
and described in this thesis.

\subsubsection{Other Developments}

An extension to \gls{zmath} has started being developed by Fellar
\cite{zmathmaster}, \cite{ozmathconference} which takes the concept of
\gls{zmath} and adds object orientatedness to it. With this, \gls{zmath} has the
potential to translate not only Z specifications but object-Z specifications as
well. 

This thesis presents a very basic user interface to use with \gls{zmath}.
Further developments on the user interface has been expanded during an
internship by Mihaylova \cite{zmathuser}, \cite{zmathinternship}. The expansion
on the user interface allows users to load and write their specifications. As
well as going through each of the correctness checks, viewing the various graphs
and skeletons all in one screen.

\subsection{Future Developments}
\label{subsec:futuredevelopments}

The future developments of \gls{zmath} have been discussed occasionally between
students and supervisors during meetings. This section puts together and
summarises these ideas and presents them to the reader in order to provide a
general idea of future developments.

\subsubsection{Automisation of the annotation}
\label{subsubsec:automisation}

At present, the user needs to annotate their formal specification by hand using
\LaTeX{} commands before being check by the various correctness checkers. This
sometimes can be a time-consuming task especially if the user isn't familiar to
\LaTeX{} syntax. An advancement on this would be if the user would be able to
visually see the Z specification as schema boxes (such as the compiled version
of \LaTeX{}) and then drag and highlight using mouse and buttons to annotate the
specification with \gls{zcga} colours and \gls{zdra} instances. This idea could
be done in a similar way to the annotations done in the original \gls{math}.
Another way to ease the users input is if the labels would automatically label
what user input. For example if the user labelled the variable `\emph{v?}' as a
term then all other variables `\emph{v?}' would also be labelled a term
automatically. This way the user wouldn't need to repeat the labels they have
already done. This would drastically increase the workload for the user
especially on very large specifications.

\subsubsection{Extension to more complex proof obligations}
\label{subsubsec:extenproofobli}

The proof obligations described in this thesis are properties to check the
consistency of the specification. The current proof obligations for Z
specifications are to give a flavour of what kind of properties to prove about
the system and to ease the user in proving these properties. As mentioned before
proof obligations for formal specifications is indeed a research subject in it's
own right and more complex proof obligations can be developed to work alongside
the \gls{zmath} framework. These proof obligation can come into the \gls{gps}
part of the translation and follow through to the complete proof. If there are
hint's or simple proof tactics to prove these properties then they can also be
added to step 6 which would allow the user to get an idea of how to finish of
the proofs.

\subsubsection{Any formal specification to any theorem prover}
\label{subsubsec:anyformanythe}

This thesis describes how the \gls{zmath} framework can translate a Z
specification into the theorem prover Isabelle. However, there are many other
theorem provers which are preferred by certain users and ultimately the
\gls{zmath} framework should be able to translate from the \gls{gps} into a
theorem prover of the users choice and not just be restricted to Isabelle. In
this case steps 1 to 4 would be the same, regardless of which theorem prover the
user wishes to translate to. The change would be made in step 5 when creating a
skeleton of the specification in the chosen theorem prover. Other theorem
provers which \gls{zmath} could translate to would be
Mizar/HOL-Z/ProofPower-Z/Coq etc.

There are many other formal languages to write specifications in which could be
another idea for future research. \Gls{zmath} currently parses through Z
specifications however, further research could be done for \gls{zmath} to work on
any formal language such as alloy, event B, UML or VDM. Investigation on whether
the grammatical categories in the \gls{zcga} or instances in the \gls{zdra}
would need adapting. Otherwise the current annotations would be suitable for any
formal notation and only the implementation would need to be changed.

\subsubsection{Informal specifications}
\label{subsubsec:informalspecs}

A final future idea would be to combine parts of \gls{math} which handles
mathematical documents written in part mathematics and part english and to
translate informal specifications into theorem provers. With this idea, perhaps
a TSa aspect would need to be adapted for informal specifications. So that a
system specification written completely in english could be checked for
\gls{zcga}, \gls{zdra} and ultimately translated fully into a theorem prover.

\subsubsection{More than one system specification in one document}

Occasionally, some systems are made up of lots of smaller subsystems. Or
 one may want to design a system specification which are unrelated to
each other in one single document. Currently the \gls{zcga} in the \gls{zmath}
supports this. This is because the \gls{zcga} checker first goes through the
annotated specification and adds all correct categories into sets e.g correct
terms\footnote{By correct terms we mean terms which are labelled with \gls{zcga}
annotations such as $\backslash$\texttt{term} and are weakly correctly typed.}
go into a python list called `\texttt{correct\_terms}' and all correct sets go
into a python list called `\texttt{correct\_sets}' etc. 

%\footnote{By correct terms we mean terms which are labelled with
%\verb|\term{..}| and are weakly correctly typed.}

If the program reads the line \verb|\specification{}|, which denotes a new
specification, all these python lists a reset and the weak type checker starts
again.

However the \gls{zdra} does not have the ability to check multiple
specifications e.g create many separate dependency graphs or multiple \gls{gps}.
If there are more than two specifications in a document and they both contain
the same instance name (e.g \texttt{SS1}) then the \gls{zdra} checker will
regard this as the same instance and will ask to rename one of them. For future
work, it may be ideal to do something similar. We can add the a function in the
\gls{zdra} program to reset all instances and relationships if it sees a new
specification or `\verb|\theory|'.

\section{Conclusion}
\label{sec:conclusion}

This thesis presents an approach to translate a formal specification into a
theorem prover in a step by step fashion. This new approach is aimed at novices
at theorem proving which could learn by example on how to translate
specifications. Proving the properties themselves is still a difficult task but
a large chunk of the work is done already automatically by \gls{zmath}. By
checking a system specification within a theorem prover adds a level of rigour
to the planned system and therefore adds a degree of safety. Perhaps one day
there will be a system which can parse through a specification written in
natural language with diagrams and tell the user automatically if it is all
correct and all conditions are satisfied. Perhaps one day, we will have systems
with no bugs at all.