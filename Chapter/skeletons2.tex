\chapter{General Proof Sketch aspect and beyond}
\label{chap:gpsa2isa}

When translating from the \gls{zdra} annotated specification to the Isabelle
skeleton, the syntax needed to be changed in order for the specification to
parse through Isabelle. In this section we outline how the Z specification is
translated into Isabelle.

If the user has labelled a theory in the specification (T\#) then that will
begin writing an Isabelle skeleton.

For example if we had an empty specification and named it "a" then the program
will create an empty Isabelle skeleton such as:

\begin{verbatim}
theory gpsa_a
imports
main
begin
end
\end{verbatim}

If the user labels a schema `\texttt{SS1}' meaning the stateSchema of the
specification then that in Isabelle becomes a `\texttt{record}' and a
`\texttt{locale}' is created. Using our example of specification named
`\texttt{a}' we get the following (after the premuable described before):
\begin{verbatim}
record SS1 =
(*DECLARATIONS*)
locale a =
fixes (*GLOBAL DECLARATIONS*)
assumes SI#
begin
end
end
\end{verbatim}

If there are no state invariants in the state schema at this point then there is
no `\texttt{assumes SI\#}') line.

All other schemas including changeSchemas, outputSchemas, preconditions that are
schemas, post conditions that are schemas and all other state schema become
definitions in the Isabelle skeleton. So for example if we have the following
schema write in \gls{zdra}
\begin{verbatim}
\draschema{CS1}{
\begin{schema}{b}
someDeclaration
\where
\draline{PO1}{someExpression}
\end{schema}}
\end{verbatim}

Then when translating into the Isabelle skeleton it becomes:
\begin{verbatim}
definition CS1 ::
"(*CS1_TYPES*) => bool"
where
"CS1 (*CS1_VARIABLES*) == PO1"
\end{verbatim}

At this stage it doesn't matter what the declarations and expressions are as
they get filled in at the next stage. The Isabelle skeleton only uses the
\gls{zdra} labels to be created.

Totalising schemas, written either horizontally or vertically in a specification
become definitions when translating into the Isabelle skeleton. For example if
we have the following totalisingSchema:

\verb|\draline{TS1}{someSchema == someExpression}|

This would translate to the Isabelle skeleton as:

\begin{verbatim}
definition TS1 ::
"(*TS1_TYPES*) => bool"
where
"TS1 (*TS1_VARIABLES*) == (*TS1_EXPRESSION*)"
\end{verbatim}

Again, at this stage it doesn't matter what the expression is. As it gets filled
in at the next stage.

\section{Proof Obligations in Isabelle Syntax}

Lemmas which are proof obligations. That is instances with the a \gls{zdra} name
\texttt{L\#\_CS\#} where `\texttt{\#}' is a number become \texttt{lemma's} in
Isabelle syntax. The translation from the \gls{gpsa} into Isabelle syntax
depends if the changeSchema in question has a precondition, postcondition or
both. We use definition \ref{defa} in aid with the translation.

\subsection{Proof Obligation translation where the schema has a precondition}

If the changeSchema in which the proof obligation is about has a precondition as
well as a postcondition then the translation will be as follows. 

If an instance has the \gls{zdra} name `\texttt{L1\_CS1}' and we have the
relations (CS1, requires, PRE1), (PRE1, allows, PO2) and (CS1, uses, IS1) then
the Isabelle skeleton syntax would be as follows:

\begin{verbatim}
lemma L1_CS1:
" \<exists> (*CS1_variables :: CS1_TYPES*).
(PRE1)
\<and> (PO2)
\<longrightarrow> ((SI1)
\<and (SI1'))"
sorry
\end{verbatim}

We use the Isabelle word `\texttt{sorry}' to tell the theorem prover to skip a
proof-in-progress and to treat the goal under consideration to be proved. This
then causes the Isabelle skeleton to be an incorrect document but is a goal the
user may prove at a later stage after the skeleton has been filled in.

If the instance in the \gls{gpsa} was `\texttt{L1\_CS1}' and the relationship
only had a precondition and no post condition ie (CS1, requires, PRE1) and (CS1,
uses, IS1) but not the allows relationship the syntax in the Isabelle skeleton
would be 

\begin{verbatim}
lemma L1_CS1:
" \<exists> (*CS1_variables :: CS1_TYPES*).
(PRE1)
\<longrightarrow> ((SI1)
\<and (SI1'))"
sorry
\end{verbatim}

Where SI1 is the stateInvariants used in the stateSchema and SI1' is the
stateInvariants prime.

\subsection{Proof Obligation translation where the changeSchema has only postcondition}

If the instance in the \gls{gpsa} was L1\_CS1 and CS1 only required a
postcondition with no precondition that is had the relation (CS1, requires, PO2)
and (CS1, uses, IS1) then the syntax in the Isabelle skeleton would be as
follows:

\begin{verbatim}
lemma L1_CS1:
" \<exists> (*CS1_variables :: CS1_TYPES*).
(PO2)
\<longrightarrow> ((SI1)
\<and (SI1'))"
sorry
\end{verbatim}

Where PO2 is the postcondition the changeSchema requires, SI1 is the
stateInvariants in the stateSchema and SI1' is the stateInvariants prime. 

%\section{Benefits}

%The Isabelle skeleton allows for incomplete specifications to also be automated
%into a half-baked proof. This way the user can have a general outline for their
%specification and fill in the missing information directly into the Isabelle
%skeleton at a later stage. For this reason it is good to have an Isabelle
%skeleton before the full filled in half-baked proof. The user then has an
%outline if they wish to add to the specification directly as they will have an
%example of how the instances should be translated. An example of a semi formal
%specification which has been translated to step 5 (the half baked proof) is
%shown in appendix \ref{app:semiform}. Although this specification is not fully
%formal its grammatical correctness and rhetorical correctness can be checked. It
%can also start being translated into Isabelle syntax. The dependencies are shown
%between the current instances and the user can see what other information needs
%to be added for the specification to be complete.

\section{ZCGa specification to Fill in the Isabelle Skeleton}
\label{sec:zcga2fillin}

Since translating using ZMathLang to translate Z specifications into an Isabelle
skeleton can even been done on incomplete specifications, it is important to
note that if some missing information is missing e.g. a declaration, expression
etc then the comments of where these should go will not be changed. For example
if we have the line \verb|"(*CS1_TYPES*) => bool"| in the skeleton and the
schema CS1 has no declarations yet then the line will not be changed, and it is
up to the user to input the variables and the types of that definition.

It is important to note that all the \gls{zcga} annotations at this stage
disappear as the labelled information is automatically put into Isabelle syntax.

\subsection{Types and Freetypes in Z}

The program which fills in the Isabelle skeleton goes through the entire
specification and adds any Z declared types and freetypes before the record. For
example, if a specification has the following:
\begin{verbatim}
\begin{zed}
[STUDENT]
\end{zed}
\end{verbatim}

Then the line \verb|typedecl STUDENT| will be added after the first begin in the
skeleton.

If the specification had the following freetype:
\begin{verbatim}
\begin{zed} 
REPORT ::= ok | already\_known | not\_known
\end{zed}
\end{verbatim}

Then again, in the same place as the Z-Types the line
\begin{verbatim}
datatype  REPORT = ok | already_known | not_known
\end{verbatim}
is added to the skeleton.

\subsection{Declarations}

In Isabelle the types and variables are added separately. For instance if we had
the following schema:

\begin{verbatim}
\draschema{OS1}{
\begin{schema}{ab}
d: \power COLOUR
c: COLOUR
\where
\draline{PO1}{c \in d}
\end{schema}}
\end{verbatim}

Then the Isabelle skeleton for this schema will be as follows:

\begin{verbatim}
definition OS1 ::
"(*OS1_TYPES*) => bool"
where
"OS1 (*OS1_VARIABLES*) == (PO1)"
\end{verbatim}

Since we have two declarations, the filling in program would change the
definition in the skeleton as follows:

\begin{verbatim}
definition ab ::
"COLOUR set => COLOUR => bool"
where
"ab d c == (c \<in> d)"
\end{verbatim}

Therefore, from the declarations, the types replace the line
\verb|(*OS1_TYPES*)| and the variables replace the line
\verb|(*OS1_VARIABLES*)|.

\subsection{Expressions}

Since the majority of the syntax for expressions is very similar to the syntax
in Isabelle, the expressions are put in directly with minor changes. The
expressions replace the \gls{zdra} labellings.

Using our previous example shown in the last section, we have the schema
`\texttt{ab}', in the skeleton we have a label `\texttt{PO1}' which is then
replaced by the expression \verb|c \<in> d|. Note this expression is very
similar to the expression in \LaTeX{} \verb|c \in d| apart from the symbol
\verb|\in| becomes \verb|\<in>|. Table \ref{tab:latextoisabelle} shows the rest
of these automatic changes of the syntax made from \LaTeX{} to Isabelle.

{\def\arraystretch{0.5}\tabcolsep=0.5pt
\begin{longtable}[H]{|c | c | c |}
\hline
\textbf{Syntax in Z} & \textbf{Syntax in \LaTeX} & \textbf{Syntax in Isabelle}
\\
\hline
\hline
$\{...\}$ & \verb|\{...\}| & \verb|{...}|\\
\hline
$\limg...\rimg$ & \verb|\limg...\rimg| & \verb|``| \\
\hline
$\langle...\rangle$ & \verb|\langle...\rangle| & \verb|\<langle>...\<rangle>| \\
\hline
$\# $ A & \verb|\#| & \verb|card| \textit{if A is set}, \\
& & \verb|length| \textit{if A is list} \\
\hline
$\cup $ & \verb|\cup| & \verb|\<union>| \\
\hline
$\cap$ & \verb|\cap| & \verb|\<inter>| \\
\hline
$\cross$ & \verb|\cross| & \verb|\<times>| \\
\hline
$\setminus$ & \verb|\setminus| & \verb|-| \\
\hline
$\geq$ & \verb|\geq| & \verb|\<ge>| \\
\hline
$\leq$ & \verb|\leq| & \verb|\<le>| \\
\hline
$\lhd$ & \verb|\lhd| & \verb|\<lhd>| \\
\hline
$\rhd$ & \verb|\rhd| & \verb|\<rhd>| \\
\hline
$\nrres$ & \verb|\nrres| & \verb|\<unlhd>| \\
\hline
$\ndres$ & \verb|\ndres| & \verb|\<unrhd>| \\
\hline
$\implies$ & \verb|\implies| & \verb|\<Longrightarrow>| \\
\hline
$\iff$ & \verb|\iff| & \verb|\<Longleftrightarrow>| \\
\hline
$\notin$ & \verb|\notin| & \verb|\<notin>| \\
\hline
$\in$ & \verb|\in| & \verb|\<in>| \\
\hline
$\subset$ & \verb|\subset| & \verb|\<subset>| \\
\hline
$\subseteq$ & \verb|\subseteq| & \verb|\<subseteq>| \\
\hline
$\land$ & \verb|\land| & \verb|\<and>| \\
\hline
$\lor$ & \verb|\lor| & \verb|\<or>| \\
\hline
$\lnot$ & \verb|\lnot| & \verb|\<not>| \\
\hline
$\neq$ & \verb|\neq| & \verb|\<noteq>| \\
\hline
$a \mapsto b$ & \verb|a \mapsto b| & \verb|(a,b)| \\
& & \verb|` '| \textit{if set preceding using} \verb|\<rightharpoonup>| \\
\hline
$\power A$ & \verb|\power A| & \verb|A set| \\
\hline
$\nat$ & \verb|\nat| & \verb|nat| \\
\hline
$\nat_1$ & \verb|\nat_1| & \verb|nat| \\
\hline
$\num$ & \verb|\num| & \verb|num| \\
\hline
$A \pfun B$ & \verb|A \pfun B| & \verb|(A \<rightharpoonup> B)| \\
\hline
$A \fun B$ & \verb|A \fun B| & \verb|(A * B) set| \\
\hline
$A \rel B$ & \verb|A \rel B| & \verb|(A * B) set| \\
\hline
$\seq A$ & \verb|\seq A| & \verb|A list| \\
\hline
$\iseq A$ & \verb|\iseq A| & \verb|A list| \\
\hline
$\seq_1 A$ & \verb|\iseq_1 A| & \verb|A list| \\
\hline
$\dom A$ & \verb|\dom A| & \verb|Domain A| \\
& & \verb|dom| \textit{if set preceding using} \verb|\<rightharpoonup>| \\
\hline
$\ran A$ & \verb|\ran A| & \verb|Range A| \\
& & \verb|ran| \textit{if set preceding using} \verb|\<rightharpoonup>| \\
\hline
$\exists$ & \verb|\exists| & \verb|\<exists>| \\
\hline
$\forall$ & \verb|\forall| & \verb|\<forall>| \\
\hline
$@$ & \verb|@| & \verb|.| \\
\hline
$R\inv$ & \verb|R\inv| & \verb|\<R^-1>| \\
\hline
$R^{k}$ & \verb|R^{k}| & \verb|\<R^k>| \\
\hline
\caption{A table showing the symbols which are changed from Z specifications in \LaTeX{} to Isabelle.}
\label{tab:latextoisabelle}
\end{longtable}
}

Some symbols in the Z toolkit are the same regardless of whether they are a
sequence/list/set or total function/partial function. However the syntax
sometimes varies in Isabelle on these occasions.

One part of the Z mathematical toolkit which we need to rewrite are the use of
partial functions. In Isabelle/HOL all functions are total therefore 
We translate total functions as a set of pairs \verb|(A * B) set|, we use the
\verb|\<rightharpoonup>| symbol to describe partial functions in the isabelle syntax. 
%Any proofs to check for partial functions if needed can be done by the user in step 6 shown in
%figure \ref{fig:steps} but the details of these proofs are beyond the scope of
%this thesis.
%but there are ways to
%do certain proofs about partial functions \cite{Krauss08definingrecursive}.%

In the following section we highlight the symbols in the Z toolkit in {\color{set}red}
and the corresponding Isabelle translation in {\color{blue}{blue}
Other parts which differ are the following:

\begin{itemize}
\item {\color{red}`\verb|\#|'} is a symbol which takes a set or list \footnote{by list we
mean an ordered set} as a parameter and returns the number of elements within
that set or list. In Z, {\color{red}`\verb|\#|'} can be used for both set and list, however
in Isabelle we must translate {\color{red}`\verb|\#|'} into {\color{blue}`\verb|card|'} if the parameter is
a set or {\color{blue}{`\verb|length|'}} if the parameter is a list.

\item {\color{red}`\verb|\mapsto|'} is a symbol which takes two terms and returns a term. If
the {\color{red}`\verb|\mapsto|'} term is in a total function where the Z syntax is {\color{red}{`\verb|A \fun B|'}} then the 
{\color{red}`\verb|\mapsto|'} term will be translated as {\color{blue}{`\verb|(f,s)|'}} in isabelle. However if the
{\color{red}`\verb|\mapsto|'} term is in partial function set then the {\color{red}{`\verb|\mapsto|'}} term will be
translated as {\color{blue}{`\verb|f s|'}}.
For example
if {\color{red}`\verb|funset: A \fun B |'} and {\color{red}`\verb|c \mapsto d \in funset|'}
then the isabelle translation will be {\color{blue}{$(c,d) \in funset$}}
however if {\color{red}{`\verb|pfunset: A \pfun B |'}} and {\color{red}{`\verb|c \mapsto d \in pfunset|'}}
then the isabelle translation will be {\color{blue}{$c\ d \in pfunset$}}.

\item {\color{red}{`\verb|\dom|'}} and {\color{red}{`\verb|\ran|'}} in Z syntax are the same regardless whether it is
being applied to an element in a partial function or a total function. However
in Isabelle the syntax varies between the two types of functions. If the
{\color{red}{`\verb|\dom|'}} being applied to an element {\color{red}{`\verb|k|'}}, with type
{\color{red}{`\verb|A \fun B|'}} then the Isabelle translation would be {\color{blue}{`\verb|Domain k|'}} whereas if
{\color{red}{`\verb|k|'}} has type {\color{red}{`\verb|A \pfun B|'}} then it will be
translated as {\color{blue}{`\verb|dom k|'}}. Similarly we would
have {\color{blue}{`\verb|Range k|'}} and {\color{blue}{`\verb|ran k|'}} for total and partial functions of
`\verb|k|' respectively.
For example
if {\color{red}{`\verb|funset: A \fun B |'}} and {\color{red}{`\verb|\dom k \in funset|'}}
then the isabelle translation will be {\color{blue}{$(A*B) set$}} and {\color{blue}{$Domain k \in funset$}}.
however if {\color{red}{`\verb|pfunset: A \pfun B |'}} and {\color{red}{`\verb|\dom k \in pfunset|'}}
then the isabelle translation will be {\color{blue}{$A \rightharpoonup B$}} and {\color{blue}{$dom\ k$}}.

\end{itemize}

\subsection{Schema Names}

The Names of the Schema are added to the skeleton by using the \gls{zdra} name.
For example if the specification had the line
\verb|\draschema{TS1}{\begin{schema}{ab}{..| then anywhere `\texttt{TS1}' is
listed in the skeleton it will be converted to `\texttt{ab}'. This is done
throughout the entire skeleton.

\subsection{Proof Obligations}

Using the birthdaybook specification an example we can see that we have the
following schema:

\begin{verbatim}
\draschema{CS1}{
\begin{schema}{AddBirthday}
\text{\Delta BirthdayBook} \\
\text{\declaration{\term{name?}: \expression{NAME}}} \\
\text{\declaration{\term{date?}: \expression{DATE}}}
\where
\draline{PRE1}{\text{\expression{\term{name?} \notin 
\set{known}}}}\\
\draline{PO3}{\expression{\set{birthday'} =
\set{\set{birthday} \cup \set{\{\term{\term{name?} \mapsto 
\term{date?}}\}}}}}
\end{schema}}
\uses{CS1}{IS1}
\requires{CS1}{PRE1}
\allows{PRE1}{PO3}
\end{verbatim}

The schema itself is represented in the filled in Isabelle syntax as:

\begin{verbatim}
definition AddBirthday :: 
"(NAME set) \<Rightarrow> NAME \<Rightarrow> BirthdayBook 
\<Rightarrow> BirthdayBook \<Rightarrow> DATE \<Rightarrow> 
(NAME \<rightharpoonup> DATE) => bool"
where 
"AddBirthday known' name birthdaybook birthdaybook' 
date birthday' ==
 (name \<notin> known)
\<and> (birthday' = birthday \<union> (name,date))"
\end{verbatim}

Then the proof obligation which checks that the before state and after state of
this changeSchema complies with the stateInvariants in represented as the
following in the Isabelle Skeleton:

\begin{verbatim}
lemma CS1_L1:
"(\<exists> (*CS1_VARIABLESANDTYPES*).
(PRE1)
\<and> (PO3)
\<longrightarrow> ((SI1)
\<and (SI1'))""
sorry
\end{verbatim}

This lemma filled in becomes the following proof obligation:

\begin{verbatim}
lemma AddBirthday_L1:
"(\<exists> known' :: (NAME set).
\<exists> name :: NAME.
\<exists> birthdaybook :: BirthdayBook.
\<exists> birthdaybook' :: BirthdayBook.
\<exists> date :: DATE.
\<exists> birthday' :: (NAME \<rightharpoonup> DATE).
(name \<notin> known)
\<and> (birthday' = birthday \<union> (name,date))
\<longrightarrow> ((known = dom birthday)
\<and> (known' = dom birthday')))"
\end{verbatim}

\section{Filled in Isabelle Skeleton to a Full Proof}
\label{sec:isa2ful}

The final step to get from a half-baked proof into a full proof is labelled as
step 6 in Figure \ref{fig:steps}, this is also named fill in 2. Technically the
specification the user automatically generates in fill in 1 is fully formalised
in Isabelle if there are no other properties to be proved. If the specification
is not fully formalised, using the half-baked proof generated in step 5, the
user then adds any safety properties about the specification they wish to prove
in the form of \emph{lemmas}. As the properties will be specific to the user
and/or specification it is difficult to automate this step. Therefore some
theorem prover knowledge may be required for step 6. Some of the automated
theorem prover tools such as Sledgehammer \cite{sledgehammer} may be useful when
proving the properties.

Figure \ref{fig:exampleproof} shows an example of a proof obligations generated
by \gls{zmath} proved in Isabelle. Again we have highlighted the user input in
{\color{red}red}. To help prove these properties wed have used sledgehammer
\cite{sledgehammer} which is part of the Isabelle/Hol package. The full proof of
this specification can be found in appendix \ref{app:moduleregfullproof}. At
this point, proving properties in a theorem prover may require some expertise in
the field. Proving tools such as \gls{smt} solvers
\cite{DeMoura:2011:SMT:1995376.1995394} such as sledgehammer are an interesting
area of research on it's own however these such details are out with the scope
of this thesis.

\begin{center}
\begin{figure}[H]
\centering
\begin{footnotesize}
\begin{BVerbatim}
    [commandchars=+\[\]] lemma AddStudent_L2: "(\<exists>
degModules:: MODULE set. \<exists> students :: PERSON set. \<exists> taking ::
(PERSON * MODULE) set. \<exists> p :: PERSON. \<exists> degModules':: MODULE
set. \<exists> students' :: PERSON set. \<exists> taking' :: (PERSON * MODULE)
set. ((students' = students \<union> {(p)}) \<and> (degModules' = degModules)
\<and> (taking' = taking)) \<longrightarrow> ((Domain taking \<subseteq>
students) \<and> (Range taking \<subseteq> degModules) \<and> (Domain taking'
\<subseteq> students') \<and> (Range taking' \<subseteq> degModules')))"
[+color[red]by blast]
\end{BVerbatim}
\end{footnotesize}
\caption{\label{fig:exampleproof} An example of a proof completed by user input.}
\end{figure}
\end{center}

\section{Conclusion}

This chapter described the final steps in computerising a formal specification
into full proof. It demonstrates how the program uses the automatically
generated general skeleton to  create an Isabelle skeleton. From the Isabelle
skeleton the user can then automatically fill in the skeleton using the
\gls{zcga} annotated specification giving a \gls{half}. The last step would be
to fill in any missing proofs in the \gls{half} (as show in the proof obligation
in figure \ref{fig:exampleproof}), this is still a difficult step
and may require some theorem prover knowledge however this part is difficult to
automate as different system specifications have different properties users wish
to prove, therefore tools such as sledgehammer \cite{sledgehammer} may be useful
at this point. We will demonstrate how to go from a Z specification to a proven 
specification in Isabelle in chapter \ref{ch:fullexample}.

In the next chapter we demonstrate a full example of a specification being
taken through all the steps of the ZMathLang framework.
