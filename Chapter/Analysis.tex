\chapter{Analysis}
\label{ch:analysis}

The concept for this thesis was to develop a new method to enable a stepwise
approach to prove formal specifications instead of proving specifications
in one step. The vending machine example has been fully proved
in \gls{ppz} and the birthday book example has been fully proved in \gls{hol} in
one step.
There have not been any proofs for our examples already done in Isabelle (apart
from the proofs I have already written myself which would be unfair to compare).
We will now compare these examples to the proofs done in a
stepwise method using ZMathLang.

It is important to note, the way the specifications are translated into
Isabelle/Hol syntax is just one way. There are various other ways one may choose
to translate specifications into Isabelle. Other variations are
described in \cite{Kolyang1996}, \cite{Kolyang86towardsa} and \cite{hol-z}.

Note: we use number of lines to compare the lengths of the proof and not
to measure the efficiency  or superiority over one proof over another.

To compare proofs we have looked at the amount of complexity in the specification.
For example the vending machine example uses integers as `types' in the specification.
Where as other examples such as the birthdaybook declare its own `type' within
the specification and is used throughout. The complexity of proofs are discussed more in the
next chapter.
This chapter focusses on what `expertise' is needed to create a proof for a specification
using different theorem provers.

In figure \ref{fig:propertyproof} to prove the proof obligations (red parts) one must have some
theorem prover expertise. However, helpful automated theorem proving tools which exist such as 
`sledgehammer' or `blast'. These tools are also available to a theorem proving expert. Therefore
effort is the same.
The difference is in translating the specification into Isabelle from \LaTeX. One can 
translate the specification by hand-rewrite the entire specification again into Isabelle syntax
or one can use ZMathLang for the translation with doing ZCGa and ZDRa checks a long the way.
This allows the user to `digest' the specification and moreover check that the specification isabelle
ZCGa and ZDRa correct before inputting into the theorem prover. This does not save time but 
allows checks to be done at an earlier phase instead of directly translating the entire specification
into a theorem prover and finding grammatical and logical errors at the end. The user will then
need to go over the entire specification and find where those errors have occurred.

\section{Vending Machine Example}

The vending machine example shown in appendix \ref{app:vm} is a simple
specification using only natural numbers as variables and there are no other
types in the specification. 

\begin{table}[H]
\begin{center}
\begin{tabular}{| l || l | l | l | l |}
\hline
\textbf{Method} & \textbf{expertise} &  \textbf{input} & \textbf{lines of proof}
\\
& \textbf{required} & & \textbf{for first lemma (fl)}  \\
& & & \textbf{entire proof (ep)} \\
\hline
One step into & much & Either ascii or & fl = 19  \\
\gls{ppz} & & windows extended & ep = 140 \\
& & characters & \\
\hline
Multiple steps & little & \LaTeX{} partially & fl = 3  \\
using \gls{zmath} & & automated & ep = 124 (63 automated) \\
& & into Isabelle & \\
\hline
\end{tabular}
\end{center}
\caption{The vending machine proof using \gls{ppz} versus the \gls{zmath} proof.}
\label{tab:comparevm}
\end{table}

Table \ref{tab:comparevm} shows a comparison between the vending
machine proof done in \gls{ppz} \cite{pp} and the vending machine proof done
using the \gls{zmath} method (see appendix \ref{app:vm}). To calculate the lines
of proof, all comments and empty lines have been removed from the proof and only
the content is left. Although the syntax of the proof can differ depending on
the author, for example some of the tactics can be put on a single line or can
be put on two separate ones, the lines of proof give a rough estimate in the
size of  proof.

The entire proof using the \gls{zmath} method is 124 lines, 63 of those lines
are automatically generated using the annotated \LaTeX{} document (79 lines).
This means that 50.8\% of the proof is already automatically generated without
the user having any knowledge of the theorem prover they are using. The actual
amount of lines in both the proofs are somewhat similar (140 lines compared with
124).

\begin{table}[H]
\begin{center}
\begin{tabular}{| l | l | l |}
\hline
\textbf{Type of} & \textbf{one step} & \textbf{multi step} \\
\textbf{expertise needed} & \textbf{into \gls{ppz}} & \textbf{using \gls{zmath}}
\\
\hline
\hline
Knowledge of Z &  yes & yes \\
\hline
Knowledge of theorem prover & much & little \\
\hline
Knowledge of \LaTeX & some & yes \\
(including Z symbols) & (optional) & \\
\hline
Knowledge of how to & & \\
input specification into & yes & no \\
theorem prover & & \\
\hline
\end{tabular}
\end{center}
\caption{Expertise needed for one step proof in \gls{ppz} and multi step proof using \gls{zmath}.}
\label{tab:expertise}
\end{table}

The expertise needed to do either proof is shown in table \ref{tab:expertise}.
Here we explain the different types of expertise needed in order to get the
vending machine specification into a full proof using one step or using many
steps.

\subsection{Knowledge of Z}
For both methods the user will need to have some form of Z specification
knowledge. Using the \gls{zmath} method, the user also annotates the plain
specification which would then in turn allow others (such as staff in the
project team, software developers, etc) also understand the Z specification.
Both methods need the same amount of expertise in Z and the \gls{zmath} method
even shares some of the expertise with others looking at the documents produced.

\subsection{Knowledge of theorem prover}
In table \ref{tab:expertise}, it states that a `little' amount of theorem prover
knowledge is needed for the full proof using \gls{zmath}. This is because
the final step is to prove any lemmas that are left unproven (these lemmas have
been created from the original Z specification) and write new properties as
lemmas and prove them if needed. However the original specification is
automatically translated into your chosen theorem prover syntax and thus if the
user needs to add more parts to the specification they already have an idea of
the syntax to use. By translating the specification and proving it in one big
step the user will need to learn how to input the specification first, as well
the syntax of a specification in the chosen theorem prover language and write up
a full proof. 

\subsection{Knowledge of \LaTeX}
The translation path using the \gls{zmath} methods assumes the user already
knows how to write a Z specification using \LaTeX{}. The user then annotates
these specifications using the annotations in the \gls{zmath} style package. The
\LaTeX{} expertise required for the translation is enough so the whole Z
specification is covered. The input of the schema boxes, Z characters etc are
all imputed using the zed style package, which the user can learn using Mike
Spivey's reference card \cite{zrefcard}.

The schema boxes and symbols are written in \gls{ppz}'s own syntax. \gls{ppz}
also has a user interface, (PPXPP), which uses an extended character set
instead of ascii to input the specifications and their proofs. In this, the user
may open a palette in which they can search for the symbol they wish to use and
click on it. The same works with schema boxes, axiomatic definitions, generic
definitions etc. 

The translation method using \gls{ppz} requires some \LaTeX{} knowledge which is
optional. This is only if the user wishes to extract the formal material for
typesetting their proofs. The shell script \textbf{doctex} allows the user to
prepare a \LaTeX{} file using the \gls{ppz} extended character set. However to
typeset the proof the instructions say that some familiarity with \LaTeX{} is
required.

\subsection{Knowledge of input of specification}
Discounting the tactics and lemmas needed to prove the specification. A large
part of full proof for the specification is to input the specification itself
into the chosen theorem prover. By translating the specification in one big step
using \gls{ppz} the user must already have vast knowledge of \gls{ppz} to do
this. That is, to translate the specification itself in one big step into
\textbf{any} theorem prover requires a lot of knowledge about the chosen theorem
prover. By using the \gls{zmath} method to translate the specification itself
requires no knowledge about Isabelle by the user, as all the Isabelle syntax is
automatically translated from the annotated specification written in \LaTeX{}. 

\section{Birthday Book Example}

The birthday book example (shown in appendix \ref{app:bb}), was created by Spivey
\cite{spiveyreferencemanual}. This example is a specification which describes a
system of a birthday book where the main functions include adding a person and
their birthday, removing a person and their birthday etc. This system uses sets
and it's own types, \texttt{NAME} and \texttt{DATE}.

\begin{table}[H]
\begin{center}
\begin{tabular}{| l || l | l | l | l |}
\hline
\textbf{Method} & \textbf{expertise} &  \textbf{input} & \textbf{lines of proof}
\\
& \textbf{required} & & \textbf{for first lemma (fl)}  \\
& & & \textbf{entire proof (ep)} \\
\hline
 \gls{hol} & some  & automated into  & fl = 5  \\
 & & ZeTa, manually into & ep = 361 \\
& & \gls{hol} & \\
\hline
 Multiple steps &  little & \LaTeX{} partially  & fl = 8  \\
using \gls{zmath} & & automated & ep = 120 \\
&  & into Isabelle &  \\
\hline
\end{tabular}
\end{center}
\caption{The birthday book proof using \gls{hol} versus the \gls{zmath} proof.}
\label{tab:comparebb}
\end{table}

Table \ref{tab:comparebb} shows the comparison between the birthday book proof
done using Hol-Z \cite{hol-z} and the birthday book proof done using the
\gls{zmath} method (see appendix \ref{app:bb}). Again to calculate the
lines of proof, all comments and empty lines have been removed from the proof.
Since the birthday book proof in \gls{hol} comes in many different files, all
the lines from these files have been added. The translation via \gls{zmath}
translates to Isabelle using just the \texttt{Main} isabelle package.

The first lemma (fl) in the table has been calculated from the "pre addBirthday
lemma" which is called \texttt{lemma AddBirthdayIsHonest} in the \gls{zmath}
method and \texttt{zlemma lemma2} in the \emph{Rel\_Refinement.thy} file using
the \gls{hol} method. The full proof using the \gls{hol} method is 361 lines,
however this is split up into 5 files. The \emph{BBSpec.holz} which is
automatically generated using the ZeTa-to-\gls{hol} converter. This converted
consists an adapter that is plugged into ZeTa and converts the \LaTeX{}
specification into an SML-file. The \emph{BB.thy} file which is used to import
\emph{Fun\_Refinement.thy} and \emph{Rel\_Refinement.thy} and \emph{BBSpec.thy}
which is used to import the specification from the SML-file. In order to prove
the specifications in \gls{hol}, there are 17 other theory files which have been
created in order to use tactics an lemmas, these include \emph{ZSeq.thy},
\emph{Z.thy}, \emph{ZPure.thy}.

The raw \LaTeX{} file used for the \gls{hol} method is 97 lines, this is
automatically generated into an SML file which can be imported into \gls{hol}
which is 17 lines long. The raw \LaTeX{} file which is used for the \gls{zmath}
method is 96 lines which automatically generates a single theory file containing
the environment and the specification, this file is 70 lines.

\begin{table}[H]
\begin{center}
\begin{tabular}{| l | l | l |}
\hline
\textbf{Type of} & \textbf{large steps} & \textbf{small steps} \\
\textbf{expertise needed} & \textbf{into \gls{hol}} & \textbf{using \gls{zmath}}
\\
\hline
\hline
Knowledge of Z &  yes & yes \\
\hline
Knowledge of theorem prover & some & little \\
\hline
Knowledge of \LaTeX & yes & yes \\
\hline
Knowledge of how to & some & no \\
input specification into &(sml into \gls{hol})&  \\
theorem prover &  &  \\
\hline
\end{tabular}
\end{center}
\caption{Expertise needed for one step proof in \gls{ppz} and multi step proof using \gls{zmath}.}
\label{tab:expertisebb}
\end{table}

Table \ref{tab:expertisebb} shows the type and amount of expertise needed in
order to get from a specification into a fully proof.

Since starting this thesis the Hol-Z project is no longer developed. Therefore the
comparison done here is limited with the support available.

%How can we judge based on table 10.2 that it is more efficient to do
%PPZed than in MathLang?  The discussion you give in sections
%10.1.1.. 10.1.4 does not conclude the efficiency.  Reading your
%section 10.1.4, you seem to be focusing on knowledge of the language
%of Isabelle.  So here the question that poses itself is again that to
%fill the Isabelle part, an Isabelle expert is needed and the non
%Isabelle expert will neither contribute nor understand that part. So,
%what is the role of a non Isabelle expert?  Why is he ever included in
%the process?

It is important to note that the time it takes translating a Z specification into 
a theorem prover with full proofs may or may not be longer using small steps in ZMathLang Vs
using Hol-Z, as that will depend on the user and their expertise/ typing skills/
knowledge of Z etc.
However what table \ref{tab:expertisebb} does show is that for both methods the user would
need knowledge of Z and of \LaTeX{}. For the actual translation of the specification alone (without proofs) 
the user would need none to very little knowledge of the theorem prover using ZMathLang but will need some 
knowledge of the theorem prover using Hol-Z.
Using Hol-Z the user would also need to type up the specification again whereas in Z-MathLang the user
would need to annotate the specification in \LaTeX{}. The time it takes to do this again would
depend on the user and if they are comfortable with the Hol-Z tool or \LaTeX{} annotations.
Therefore the `efficiency' would be the same for both methods however using the Hol-Z method the user
would need additional Hol-Z and \LaTeX{} expertise whereas using ZMathLang the user would need to have just \LaTeX{} 
expertise on it's own. This is for the translation on it's own.

To complete the proofs for the specification the user would need both \LaTeX{} and some theorem
prover expertise. Therefore the `efficiency' is about the same.

\subsection{Knowledge of Z}

For both methods the user will need to have some knowledge of Z specifications.
This is because in both methods the initial step is to write the specification
in \LaTeX{} for the system. However by using the \gls{zmath} method when
annotating the Z specification in ZCGa, to compiled documents outputs the weak
types in different grammars. This then allows others to identify certain parts
of the Z syntax. Therefore the knowledge of Z is exactly the same in both these
methods.

\subsection{Knowledge of theorem prover}

Table \ref{tab:expertisebb} shows that by using the \gls{zmath} method `little'
knowledge of theorem prover is needed. This is because the final step to prove
any unproven lemmas and write and new safety properties and lemmas and prove
them. The user may need some theorem prover knowledge to compete this
final step. However the entire specification itself is already written in the
chosen theorem prover (in our case Isabelle) and the user does not need to
import any further definitions which are part of the original specification.
However, by using \gls{hol} method the user will need `some' theorem prover
knowledge. Although it is possible to ease the translation of the specification
into \gls{hol} using ZeTa (see section \ref{knowledgeofinputforbb}), the user
will need to have the \gls{hol} plugin knowledge as well as the original
Isabelle/Hol Knowledge to do the proofs for the specification.

The Hol-Z proof comes in 10+ files written in languages (hol-z, zeta, sml etc), the ZMathLang proof
includes:
\begin{enumerate}
    \item 1 ZCGa Labelled tex document + it's pdf output
    \item 1 ZDRa Labelled tex document + it's pdf output
    \item GPSa
    \item proof skeleton in Isabelle
    \item Filled in skeleton
    \item Full proof in Isabelle
\end{enumerate}

Both theorem provers offer a \LaTeX{} written specification.
The ZMathLang toolkit also produces graphs in order to help the user understand
the layout and dependencies of the specification. Hol-Z does not offer such diagrams and
therefore understanding the proof purely via code may be more difficult.

An e-mail from `Burkhart Wolff' (who wrote the Hol-Z tutorial) explained the following
about Hol-Z:
\\
`\texttt{The port of the library is not particularly difficult though.
What is hairy, is the front-end, and in particular the Zeta Parser.
A project around 2012 trying to port it to a more recent Isabelle-version
and a \\ new Z parser/typechecker front-end was abandoned.}'

\subsection{Knowledge of \LaTeX}

In both methods the user will need to have the same amount of knowledge of
\LaTeX{}. This is because in both cases, the user will need to input their
specification using \LaTeX{}. The only difference in this aspect is that the
user will need to annotate their specification using \gls{zmath} annotations
(ZCGa and ZDRa) in the \gls{zmath} method or the user will need to annotate
their specification using \gls{hol} annotations (proof obligations, Z sections
etc). In both cases the user will need to know how to import a package into
\LaTeX{} and then read the instructions in either case on the annotations which
need to be used.

\subsection{Knowledge of input of specification}
\label{knowledgeofinputforbb}

When translating Z specification into the \gls{hol} theorem prover there are two
ways a user can do this. The first method for convenience, involves the user
writing their specification in \LaTeX{}, using the \gls{hol} package to annotate
their specification. Then the ZeTa-to-\gls{hol} plug-in type checks the
specification and generates .holz files which can be imported by the user into
the \gls{hol} theorem prover. The method is have the user write the
specification directly into \gls{hol} circumventing ZeTa. In both these methods
the user would need at least some form of \gls{hol} prover knowledge. The latter
would need more than the former. By using the ZeTa-to-\gls{hol} plug-in, the
user can write their specification in \LaTeX{} format with the \gls{hol}
annotations (very similar to \gls{zmath} method), however the user will need to
know how to import the .holz files into the \gls{hol} theorem prover, unpack the
schemas and values, and how to write and prove the properties.

To input the specification into the chosen theorem prover using \gls{zmath} the
user will need no theorem prover knowledge at all. This is because the annotated
specification will be automatically translated into Isabelle/Hol when using the
\gls{zmath} method. The program will automatically generate an `.thy' file which
is a skeleton of the specification and then automatically fill in the
specification using the information from the ZCGa annotations.

\subsection{Comparison with similar tools for Z}
\label{subsec:provingSystemsForZ}

Hol-Z \cite{hol-z} also analysis the Z notation, and is also a
proof environment for Z. Hol-Z is embedded in Isabelle/HOL therefore it provides
a Z type checker, documentation facilities and refinement proofs with a theorem
prover. The Z specification is implemented in \LaTeX{} then typed checked using
an external plug in Zeta, it is then transformed into SML files to be added into
the Hol-Z theorem prover environment. The user will need to have some good
expertise in using the Hol-Z proof environment in order to fully prove the
specification. Hol-Z works differently to \gls{zmath} as it is a proof
environment for Z specifications and will only analyse the parts of the document
written in Z syntax, whereas \gls{zmath} will check any parts of the document
which have been annotated by the user using the \gls{zcga} and \gls{zdra}
annotations. Hol-Z uses Zeta \cite{zeta} as a Z type checker.

Zeta \cite{zeta} is an open environnement for the development, analysis and
animation of Z specifications. The system is aware of dependencies between the
units and attempt to exploit the units when they are changed. Zeta is different
to the \gls{zcga} type checker because the \gls{zcga} will read the annotations
written y the user which could include some informal text as well as Z syntax.
The \gls{zcga} did not aim to be a strong type checker like Zeta as the
\gls{zcga} intends to check the grammatical correctness of the specification at
a high level, where the weak types can be check when a specification is in the
process of being developed. The Z specification would be \underline{strongly
type checked} once the user would have translated the specification into
Isabelle (step 6 of the \gls{zmath} path.)

Fuzz \cite{spiveyfuzz} is Mike Spiveys type-checker for Z specifications.
It includes style files for \LaTeX{} and checks for the logical correctness and
Z type correctness of Z specifications. It is a strong tpe checker for Z which
takes Z specifications written in \LaTeX{} as input. This is similar to the Zeta
checker as it checks the specific "Z Types" and whether the correct symbols are
used. This is different to the \gls{zcga} type
checker as the ZCGa checks for weak types and is aimed at early soft type
checking (as full type checking is done once the specification is translated
into ISabelle). The \gls{zcga} . Therefore the grammatical correctness
of partially formal specification can also be checked. The \gls{zmath} framework
presented in this thesis uses the `zed' \LaTeX{} style package to typeset the Z
specifications in the documents.

Proofpower \cite{pp} is a suite of tools supporting specification and proof in
Higher Order Logic (HOL) and in the Z notation. Proofpower contains 6 packages:
\begin{enumerate}
    \item[PPDev] - The ProofPower developer kit, mainly comprising SLRP, a parser generator for Standard ML.
    \item[PPTex] - The ProofPower interface to TeX and LaTeX.
    \item[PPXpp] - The X Windows/Motif front-end for ProofPower.
    \item[PPHol] - The HOL specification and proof development system.
    \item[PPZed] - The Z specification and proof development system.
    \item[PPDaz] - The Compliance Tool for specifying and verifying Ada programs.
\end{enumerate}

\section{Conclusion}
This section compares 2 specifications written in Z which have been proven in a
theorem prover previously with the proofs done using \gls{zmath}. The
\gls{zmath} framework allows the user analyse their formal specification and
assists them translating the specification itself into a theorem prover. However
the last step of the framework to prove properties about the specification is
still a difficult step in both translation paths (via \gls{zmath} or via another
route). However the \gls{zmath} framework is there to give a helping hand to
users who are complete beginners in proving formal specifications. Proving the
actual properties and the proof obligations of the specification are a whole
research area on their own and beyond the scope of this thesis but touched upon
in chapter \ref{ch:background}.

The \gls{zmath} aspects and it's tools can be used as helpful means to the user digest 
the specification and allow them to understand
the syntax of their specifications.

In the next chapter we conclude our findings of this thesis and highlight what
areas are of interest for future work.

%
%
%\subsubsection{Specification Input}
%
%We compare the way the specification is imputed into the theorem prover. Also
%it may be interesting to see how many lines are needed for the first lemma. We
%could compare the entire proof however with additional comments and information
%needed for the proofs this may be more complicating to analyse.
%
%\begin{figure}[H] \begin{BVerbatim}[fontsize=\scriptsize, baseline=t]
%%EZ%%BH%%BH%%BH%%BH%%BH%%BH%%BH%%BH%%BH%%BH%
%%SZS%exact_cash%BH%%BH%%BH%%BH%%BH%%BH%%BH% %BV%   cash_tendered? :%bbN%
%%BT%%BH%%BH%%BH%%BH%%BH%%BH%%BH%%BH%%BH%%BH% %BV%   cash_tendered? = price
%%EZ%%BH%%BH%%BH%%BH%%BH%%BH%%BH%%BH%%BH%%BH% \end{BVerbatim}
%\begin{BVerbatim}[fontsize=\scriptsize, baseline=t] definition exact_cash ::
%"nat  \<Rightarrow> bool" where "exact_cash cash_tendered  \<equiv>
%(cash_tendered = price)" \end{BVerbatim} \caption{The exact\_cash schema
%written in \gls{ppz} (left) and Isabelle (right).} \label{fig:exact_cash}
%\end{figure}
%
%The input for a \gls{ppz} proof can be either in ascii or an extended character
%set. The ascii character set is to input the schemas of a Z specification into
%a normal text editor. Figure \ref{fig:exact_cash} shows the same schema
%(\emph{exact\_cash}), written both \gls{ppz} and Isabelle syntax. Note that the
%schema written in \gls{ppz} is more similar to a normal drawing of a schema
%than the Isabelle version. However with the ascii version of the Z schema it is
%difficult to distinguish between the actual words of the schema and the actual
%schema frame itself.
%
%\begin{figure}[H] \begin{center} \includegraphics
% [width=6cm]{Figures/Conclusion/exactcash.png} \caption{The exact\_cash schema
% written in \gls{ppz} using the extended character set.}
% \label{fig:exactcash_ext} \end{center} \end{figure}
%
%On the other hand, when converting a \gls{ppz} proof into the extended
%character set. The Z specification is neater and shows the proper frames for
%the schema boxes. Figure \ref{fig:exactcash_ext} shows the exact\_cash schema
%box with the extended character set. However to get to this stage the user must
%first learn how to convert between the ascii and the extended character
%versions. They must also learn how to use the \textbf{xpp} interface
%\cite{xppMan}, which combines a general purpose editor with a command interface
%for the use of \gls{ppz}.
%
%\subsubsection{Expertise required in order to get a full proof}
%
%To actual obtain the full proof of the vending machine requires a lot less
%expertise if using the MathLang method than if going straight into a \gls{ppz}
%proof. Some knowledge about Z specifications may be required so that one may
%annotate what are declarations, expressions, state schemas etc. But no theorem
%prover knowledge is needed to actual insert the specification itself into
%Isabelle syntax (step 5 in figure \ref{fig:steps}). However to start writing
%the specification into \gls{ppz} the user will first need to read the \gls{ppz}
%manual to know how to get started. If using the ascii version they must know
%how to translate schema frames into ascii code and how to run the code. If
%using the extended character set, not only does the user need to learn
%\gls{ppz} syntax but also how to start and run the \textbf{Xpp} interface as
%well as the \gls{ppz} syntax. The user must also open the palette of symbols
%(which shows all symbols which can be used in Z) and search for the symbols
%needed.
%
%By using the \gls{zmath} steps as long as the user has labelled their
%specification correctly and checked it using ZCGa and ZDRa then the actual
%specification is automatically converted into Isabelle syntax. At no point
%during this translation does the user need to know or have used Isabelle
%previously. Nonetheless, from step 5 to step 7 the user will need to at least
%familiarise themselves with their chosen theorem prover (in this case
%Isabelle). Step 6 and step 7 include adding the properties the user wishes to
%prove as lemmas and then proving them. Since this is done in \gls{ppz} as well
%then this would be no extra effort then if the user has chosen to do it in one
%steps. But since the work from step 0 to step 5 is easier in steps using
%\gls{zmath} then the overall translation in \gls{zmath} would be easier to do.
%
%\subsubsection{Proof and tactics used to prove the lemma}
%
%\begin{figure}[H] \begin{BVerbatim}[fontsize=\scriptsize, baseline=t]
%set_goal([],%SZT%pre VM1 %equiv% (0 < stock %and% cash_tendered? = price %and%
%0 %leq% takings)%>%);
%
%a (rewrite_tac [VM1, VM_sale, some_stock, VM_operation, VMSTATE, exact_cash]);
% a (pure_rewrite_tac [z_get_spec %SZT%(_ %leq% _)%>%]); a (rewrite_tac[]); a
% (REPEAT z_strip_tac); a (z_%exists%_tac %SZT%(bars_delivered! %def% 1,
% cash_refunded! %def% cash_tendered? 
%   + ~ price, stock' %def% stock + ~ 1, takings' %def% takings + price)%>% THEN
%     rewrite_tac[]); a (PC_T1 "z_library_ext" asm_rewrite_tac [rewrite_rule []
%     price]); a (LEMMA_T %SZT%stock + ~ 1 %leq% stock%>% asm_tac THEN1
%     rewrite_tac[]); a (all_fc_tac [z_%leq%_trans_thm]); a (asm_rewrite_tac
%     []); a (strip_asm_tac (z_get_spec %SZT%price%>%)); a (all_fc_tac
%     [z_%bbN%_plus_thm]); val pre_VM1_thm = save_pop_thm "pre_VM1_thm";
%     \end{BVerbatim} \begin{BVerbatim}[fontsize=\scriptsize, baseline=t] lemma
%     pre_VM1: "(\<exists> stock' takings' cash_refunded bars_delivered. VM1
%     cash_tendered stock takings stock' takings' cash_refunded bars_delivered)
%     \<longleftrightarrow> (0 < stock) \<and> (cash_tendered = price) \<and> (0
%     \<le> takings)"
%
%apply (unfold VM1_def exact_cash_def some_stock_def VM_sale_def) apply auto
% done \end{BVerbatim} \caption{The exact\_cash schema written in \gls{ppz}
% (left) and Isabelle (right).} \label{fig:lemma1proof} \end{figure}
%
%We will now compare a lemma and it's proof inputted in both \gls{ppz} and
%Isabelle. Figure \ref{fig:lemma1proof} shows the same lemma and it's proof
%written in both \gls{ppz} (left) and Isabelle (right). The lemma and it's proof
%are both for the vending machine example. We can imagine that the vending
%machine specification is already in putted in the theorem prover before this
%lemma is added. We can see that the lemma itself is slightly longer written in
%Isabelle then it is in \gls{ppz}. However the proof of this lemma is a lot
%longer in \gls{ppz} then it is in Isabelle. Note that at the end of the proof in
%\gls{ppz} we have the following line: \begin{verbatim} val pre_VM1_thm =
%save_pop_thm "pre_VM1_thm";. \end{verbatim} This saves the lemma and names it
%"pre\_VM1\_thm" whereas in Isabelle we have the lemma saved as "pre\_VM1"
%automatically. As this lemma and it's proof are part of step 6 and 7 of figure
%\ref{fig:steps} which are both manual input. It is both up to the user to
%acquire at least some knowledge of the targeted theorem prover in order to
%choose the lemmas and prove them. In this case writing the properties and
%proving them require the same amount of work in both scenarios, whether the
%user is proving the specification all at once or using the \gls{zmath} steps.
%However when using the \gls{zmath} path, steps 0 to 6 require less expertise
%knowledge.
%
%
%\subsection{Birthday Book Example} Spiveys birthday book is a specification
%which adds names and birthdays to a birthday book. It uses it's own types and
%sets.
%
%\begin{table}[H] \begin{center} \begin{tabular}{| l || l | l | l |} \hline
%\textbf{Method} & \textbf{expertise} &  \textbf{input}  \\
%& \textbf{required} &   \\
%\hline 1 step into \gls{hol} & much & \LaTeX{} into    \\
%&  &  ZeTa into \gls{hol}   \\
%\hline stepwise using & little & \LaTeX{} into Isabelle    \\
%\gls{zmath} & &  \\
%\hline \end{tabular} \end{center} \caption{Comparison of the vending machine
%proof using \gls{ppz} Vs the \gls{zmath} method.} \label{tab:comparebb}
%\end{table}
%
%\subsubsection{Expertise required to get into a full proof}
%
%Similarly to \gls{zmath}, \gls{hol} breaks the translation into steps. It first
%reads a \LaTeX{} specification (again similar to \gls{zmath}), is type checked
%by ZeTa type checker and then is converted into SML-files that can be loaded in
%Isabelle. The SML files which are loaded into Isabelle need all the extra
%\gls{hol} .thy files to be checked, whereas \gls{zmath} only uses Isabelle's
%Main.thy package and nothing else. Therefore in order to prove the
%specification the user must learn all the extra \gls{hol} proving syntax as
%well as Isabelle's syntax and the Zeta type checker. Whereas in \gls{zmath} the
%user will only need to learn some Isabelle syntax in order to prove the
%properties of the specification.
%
%On the other hand \gls{hol} does have an advantage in that the by using the
%holz.sty package the user can use labels to automatically generate constancy
%conditions, refinement conditions or special safety properties. Thus, removing
%the effort from the user of doing these by hand.
%
%Since \gls{zmath} breaks the translation in more steps then \gls{hol}, it also
%has the advantage of being able to translate semi-formal specifications into
%theorem provers. Such that the specification can be written partially in Z
%format and partially in natural language. Another asset of the \gls{zmath}
%method is that the files produced from this method give not only the user but
%the programmers more information about the specification. By labelling the
%specification, the user is giving information about the types in the
%specification which can aid the developer in his work. The dependency graph and
%GoTo graph produced from the ZDRa also benefit the developers as they are then
%able to view the relationships between different parts of the specification.
%This will be especially beneficially on large scale projects where the
%different parts of the specification which relate to each other are far apart
%on the specification document.
%
%\subsubsection{Input}
%
%\begin{figure}[H] \begin{BVerbatim}[fontsize=\scriptsize, baseline=t]
%(ZAbsy.Eqn("AddBirthday", ZAbsy.SchemaText
%([ZAbsy.Unary(ZAbsy.Delta,ZAbsy.NameAppl ("BirthdayBook",[])),ZAbsy.Direct
%(["name?"],ZAbsy.NameAppl("NAME",[])), ZAbsy.Direct(["date?"], ZAbsy.NameAppl
%("DATE",[]))],[ZAbsy.Test(ZAbsy.Tuple ([ZAbsy.NameAppl("name?",[]),ZAbsy.
%NameAppl("known",[])]), ZAbsy.NameAppl ("_notin_",[])),ZAbsy.Test(ZAbsy.Tuple
%([ZAbsy.NameAppl("birthday'",[]),ZAbsy. Binary(ZAbsy.Apply,ZAbsy.NameAppl
%("_cup_",[]),|Absy.Tuple([ZAbsy.NameAppl ("birthday",[]),ZAbsy.Display([ZAbsy.
%Binary(ZAbsy.Apply,ZAbsy.NameAppl ("_mapsto_",[]),ZAbsy.Tuple([ZAbsy.
%NameAppl("name?",[]),ZAbsy.NameAppl ("date?",[])]))])]))]),ZAbsy.NameAppl
%("_=_",[]))]),ZAbsy.Type(ZAbsy.Unary(ZAbsy.
%Power,ZAbsy.Signature([("birthday",ZAbsy.
%Unary(ZAbsy.Power,ZAbsy.Product([ZAbsy.
%NameAppl("NAME",[]),ZAbsy.NameAppl("DATE", [])]))),
%("birthday'",ZAbsy.Unary(ZAbsy.
%Power,ZAbsy.Product([ZAbsy.NameAppl("NAME",[]),ZAbsy.NameAppl("DATE",[])]))),("date?",
%ZAbsy.NameAppl("DATE",[]("known",ZAbsy.
%Unary(ZAbsy.Power,ZAbsy.NameAppl("NAME",[]))),
%("known'",ZAbsy.Unary(ZAbsy.Power,ZAbsy. NameAppl("NAME",[]))),("name?",ZAbsy.
%NameAppl("NAME",[]))]))))), \end{BVerbatim}
%\begin{BVerbatim}[fontsize=\scriptsize, baseline=t] definition AddBirthday ::
%"BirthdayBook => BirthdayBook =>  NAME set => (NAME * DATE) set => NAME => DATE
%=> bool" where "AddBirthday birthdaybook birthdaybook' known' birthday' name
%date ==
%(
%(name \<notin> known) \<and> (birthday' = birthday \<union> {(name, date)})
%)"
%\end{BVerbatim} \caption{The AddBirthday schema written in \gls{hol} (left),
%and \gls{zmath} (right).} \label{fig:addbirthdaydef} \end{figure}
%
%Figure \ref{fig:addbirthdaydef} shows the AddBirthday Schema from the birthday
%book automatically generated into \gls{hol} on the left and \gls{zmath} on the
%right. Although both of these are automatically generated and not user input,
%the input into Isabelle through \gls{zmath} would be a lot easier for anyone to
%understand. The \gls{hol} input is a separate theory  file and in the main .thy
%file has the line "load\_holz "BBSpec"" is added. The user may want to go
%through some functions and make changes to the original specification, to do
%this the user would need to go back to the \LaTeX{} specification and go
%through the process again, or learn the syntax of Isabelle and \gls{hol} in
%greater detail in order to add more functions.
%
