%Glossaries
\makeglossaries

% You must define terms or symbols before you can use them in the document. This is best done in the preamble. Each term is defined using:
% 
% \newglossaryentry{<label>}{<settings>}
% 
% where <label> is a unique label used to identify the term. The second argument, <settings>, is a key=value comma separated list that is used to set the required information for the term. A full list of available keys can be found in "Defining Glossary Entries" in the main glossaries user manual. The principle keys are name and description.
% 
% For example, to define the term "electrolyte":
% 
% \newglossaryentry{electrolyte}{name=electrolyte,
% description={solution able to conduct electric current}}
% 
% In the above example, the label and the name happen to be the same. In the next example, the name contains a ligature but the label doesn't:
% 
% \newglossaryentry{oesophagus}{name=\oe sophagus,
% description={canal from mouth to stomach},
%plural=\oe sophagi}

%\newacronym{<label>}{<abbrv>}{<full>}

%\gls{<label>}
%\glspl{<label>}

\newacronym{}{}{}

\newacronym{zcga}{ZCGa}{Z Core Grammatical aspect}

\newacronym{zdra}{ZDRa}{Z Document Rhetorical aspect}

\newacronym{gpsa}{GPSa}{General Proof Skeleton aspect}

\newacronym{cga}{CGa}{Core Grammatical aspect}

\newacronym{dra}{DRa}{Document Rhetorical aspect}

\newacronym{tsa}{TSa}{Text and Symbol aspect}

\newacronym{ppz}{PPZed}{Proof Power Z}

\newacronym{hol}{Hol-Z}{Hol-Z}

\newacronym{math}{MathLang}{MathLang framework for mathematics}

\newacronym{zmath}{ZMathLang}{a toolkit for checking various degrees of correctness for Z specifications}

\newacronym{asm}{ASM}{Abstract state machine}

\newacronym{gps}{Gpsa}{General Proof Skeleton aspect}

\newacronym{gpsol}{GpsaOL}{General Proof Skeleton ordered list}

\newacronym{smt}{SMT}{Satisfiability Modulo Theories}

\newacronym{utp}{UTP}{Unifying theories of programming}

\newacronym{uml}{UML}{Unified Modeling Language}

\newacronym{sil}{SIL}{Safety Integrity Levels}

\newacronym{iec}{IEC}{International Electrotechnical Commission}

\newacronym{pcc}{PCC}{proof carrying code}

%\newacronym{half}{HalfBaked}{The automatically filled in skeleton also known as the Half-Baked Proof}

\newglossaryentry{specmod}{name=specification models,
description={A detailed description of the design and the materials used to create a system.}}

\newglossaryentry{veri}{name=verification,
description={The process of establishing the truth, accuracy or validity of a system.}}

\newglossaryentry{half}{name=halfbaked proof,
description={The automatically filled in skeleton also known as the Half-Baked Proof}}

\newglossaryentry{fm}{name=formal methods,
description={Mathermatically rigorous techniques and tools for the specification, design and verification of software and hardware systems}}

\newglossaryentry{semiform}{name=semi-formal specification,
description={A specification which is partially formal, meaning it has a mix of natural language and formal parts}}

\newglossaryentry{formalise}{name=formalisation,
description={The process of extracting the essence of the knowledge contained in a document and providing it in a complete, correct and unambiguous format}}

\newglossaryentry{computerise}{name=computerisation,
description={The process of putting a document in a computer format}}

\newglossaryentry{partial}{name=partial correctness,
description={A total correctness specification [P] C [Q] is true if and only if, whenever C is executed in a state satisfying P and if the execution of C terminates, then the state in which C’s execution terminates satisfies Q}}

\newglossaryentry{total}{name=total correctness,
description={A total correctness specification [P] C [Q] is true if and only if, whenever C is executed in a state satisfying P, then the execution of C terminates, after C terminates Q holds}}


